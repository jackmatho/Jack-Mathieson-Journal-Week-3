\documentclass{article}
\usepackage[utf8]{inputenc}

\title{Jack Mathieson Journal Week 3}
\author{jack.mathieson }
\date{August 2019}

\begin{document}

\maketitle

As in week 1 this Journal is just recording certain key themes that came up in class on Friday. It will edited and changed throughout the next week before it gets committed to Github. The following nonsense is stuff that came up in class on the 16th of August, 2:00-4:00.

\textbf{Notes}:

-Vocab that came up:

CONTROLLED VOCABULARY: from what I understand it is the explicitly stated limits of a variable- what a variable is allowed to be in a data set. E.g: all the numbers in this row (this variable) will be in Degrees Celsius (and thus NOT Kelvin).

-Fun discussion on messy and clean data followed

-Dates are evil. Make sure to note the system and where possible actually spell out the date with their three-letter abbreviations rather than use a number (Jul instead of 07 for July).

-Computational Thinking: how to speak computer. They are good at repetitive things (like counting), but terrible at creativity. They are very fast, but very stupid and in need of guidance- SPECIFIC guidance. Computational thinking is the process where we look at a task, judge if a computer could do it, and then break it down into simple instructions that a computer could follow.
Step 1: Decomposition (break it down). Step 2. Algorithm Design (come up with instructions)

\textbf{Exercise}

-Try to come up with some Computational Thinking:

\textbf{...I encountered a slight problem here and asked the following...
}

-Question: isn't all data initially collected by people? So how could a computer be useful in this part?

-Fun fact, the earliest "computers" were people. But generally yes. What a computer is good at is not the initial counting- the \textit{collecting}- but analysing once it has \textit{been} collected a computer is a fantastic tool at finding patters and counting up results quickly.

\end{document}
